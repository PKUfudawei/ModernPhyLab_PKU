\documentclass[a4paper]{article}

\input{style/ch_xelatex.tex}
\input{style/scala.tex}

\lstset{frame=,basicstyle={\footnotesize\ttfamily}}

%\graphicspath{ {images/} }
\usepackage{ctex,verbatim,geometry,amsmath,tikz,float,booktabs,bm,siunitx,enumerate,cite,diagbox}
\usepackage{pifont}%\ding{192} \ding{172}
%\usepackage{graphicx}
%\geometry{a4paper, scale=0.72}
\geometry{a4paper,left=2.5cm,right=2.5cm,top=2.5cm,bottom=2.5cm}
%%%%%%%%%%%%%%%%%%%%%%%%%%%%%%%%%%%%%%%% BEGIN DOC %%%%%%%%%%%%%%%%%%%%%%%%%%%%%%%%%%%%%%%%

\begin{document}
\renewcommand{\contentsname}{目\ 录}
\renewcommand{\appendixname}{附录}
\renewcommand{\appendixpagename}{附录}
\renewcommand{\refname}{参考文献} 
\renewcommand{\figurename}{图}
\renewcommand{\tablename}{表}
\renewcommand{\today}{\number\year 年 \number\month 月 \number\day 日}
\newcommand{\refeq}[1]{\textbf{Eq.(\ref{#1})}}
\newcommand*{\circled}[1]{\lower.7ex\hbox{\tikz\draw (0pt, 0pt)%
    circle (.5em) node {\makebox[1em][c]{\small #1}};}}
    
\title{{\Huge 近代物理实验报告{\large\linebreak\\}}{\Large 实验5-3:\ 扫描电子显微镜\linebreak\linebreak}}
%please write your name, Student #, and Class # in Authors, student ID, and class # respectively
\author{\\姓\ 名:付\ 大\ 为\\
学\ 号: 1800011105\\
邮\ 箱: \url{fudw@pku.edu.cn}\\
近代物理实验 (II)\\
(2022,春季学期)\\\\
北京大学\\
物理学院\\
2018级1班}
\date{\today}
\maketitle
\newpage

%%%%%%%%%%%%%%%%%%%%%%%%%%%%%%%%%%%%%%%% ABSTRACT %%%%%%%%%%%%%%%%%%%%%%%%%%%%%%%%%%%%%%%%
\begin{center}
{\Large\bf{摘\ 要\\}}
\end{center}

扫描电子显微镜是研究物质微观形貌及组分的重要实验装置。本实验回顾了扫描电镜的基本结构、工作原理及调节办法,在此基础上探究了驱动电子透镜中的聚光镜电流对成像质量的影响。
	
具体而言,本实验考察了400$\sim$650 范围内、间隔步长50的6个聚光镜电流值对应的\SI{1}{k} 和 \SI{10}{k} 倍放大图像,用直观及半定量(频谱分析)的办法比较了图像的像质。结果表明,随电流增大,图像分辨率增大;且在聚光镜电流较大的情况下,噪声对像质有所影响。这一结果与初步的理论分析基本吻合。
\\\\
{\bf{关键词}:}\ 扫描电子显微镜(SEM),\ 电子透镜,\ 聚光镜电流,\ 分辨率

%%%%%%%%%%%%%%%%%%%%%%%%%%%%%%%%%%%%%%%% CONTENT %%%%%%%%%%%%%%%%%%%%%%%%%%%%%%%%%%%%%%%%
\newpage
\begin{center}
\tableofcontents\label{c}
\end{center}
\newpage

%%%%%%%%%%%%%%%%%%%%%%%%%%%%%%%%%%%%%%%% Introduction %%%%%%%%%%%%%%%%%%%%%%%%%%%%%%%%%%%%%%%%
\section{引言}\label{overview}%------------------------------

1900年前后,最好的光学显微镜分辨率已接近其理论极限$\sim\SI{200}{\nm}$, 这一限制源于光的衍射行为及可见光的波长下限。为了制造更高分辨率的显微镜以看清更小层次的物质,人们不得不从新的思路出发突破该瓶颈。
	

1924年前后,为了解决旧量子论中难以调和的诸多困难,德布罗意(Louis de Broglie)在其发表的博士论文 \cite{de1924recherches} 中提出了一种全新的量子化思路;德布罗意由爱因斯坦的光量子假定出发,推而广之,认为实物粒子同样具有波粒二象性,其\textit{相波}(phase wave)波长为:
\begin{equation}
	\lambda = \frac{h}{p}
\end{equation}
该关系式与光量子的动量公式$p = \frac{h}{\lambda}$完全一致。对于电子而言,在非相对论情形下,$p = \sqrt{2mE} = \sqrt{2meU}$, 其中$U$为加速电压,则有:
\begin{equation}
    \lambda \approx \sqrt{\frac{\SI{150}{V}}{U}}\si{\angstrom}
\end{equation}\label{eq:deBroglieLambda}
	
相波的假定确认了实物粒子可类比电磁波进行研究,即电子束的运动可与光束的传播类比;且据 \eqref{eq:deBroglieLambda}, 高能电子的相波波长可以显著小于可见光波长,这启示我们可以实物粒子为媒介进行显微探测,从而突破可见光的分辨极限。
	
物质的波动性随后便得到了实验证实\footnote{例如,参见\cite{thomson1927diffraction}. },并由薛定谔(E. Schrödinger)、波恩(M. Born)等人在理论上严格化。在此基础上,类比光学显微镜,考虑电子\CJKunderdot{透射},人们设计了\textit{透射电子显微镜}(TEM\footnote{TEM是transmission electron microscopy的缩写。}),这于1932年前后首次实现,随后不久便获得了高于光学显微镜的分辨率\cite{ruska1980early};利用电子在表面上的“反射”(严谨而言,是偏角充分大的各类散射),进一步设计了\textit{扫描电子显微镜}(scanning electron microscope, SEM)。
	
SEM利用\textit{电子光学}原理,以电子束代替光束对样品表面进行扫描,收集散射电子及X射线等辐射信号、样品电流信号并加以分析,从而获得样品表面结构、分析样品基本成分。
	
在SEM的发明过程中,德国物理学家Max Knoll做出了重要的探索性的工作(\textit{参见\cite{knoll1935aufladepotentiel}});在此基础上,与现代原理基本一致的扫描电子显微镜于随后的1937年由德国物理学家Manfred von Ardenne发明\cite{mcmullan1995scanning}。如今,性能最好的 SEM 分辨率已达到 \SI{.4}{\nm} 以下\supercite{Nanotech86:online}。
	
相比传统的光学显微镜,SEM 的信号来源要丰富的多,因而有着更加强大的功能。如今SEM已成为化学、生物、材料等许多领域中的重要显微、分析技术。
	
本实验关注SEM最为基本的信号与功能,即利用散射产生的二次电子(secondary electron, SE)进行样品形貌分析;选取了二次电子成像的一项关键参数,即电子聚光镜电流,探究其对成像的影响,从一较小的角度具体展现SEM的基本性能及其决定因素。

%%%%%%%%%%%%%%%%%%%%%%%%%%%%%%%%%%%%%%%% Theory %%%%%%%%%%%%%%%%%%%%%%%%%%%%%%%%%%%%%%%%
\newpage
\section{理论}\label{theory}%------------------------------
虽说电镜、光学显微镜的类比基础依赖于物质的波动性,但初步的计算分析可利用经典电磁学的方法实现;这实际上是\textit{几何光学极限}的后果,由此分析外场对电子束的作用规律即为(几何)电子光学。具体而言,类比:外加静电磁场$\approx$透镜,起聚光等作用;静电场$\sim$介质,使光线(电子束)发生偏折。此结论基于下述电磁学事实:
\begin{itemize}
\item 带电粒子在等电势区内沿直线运动;
\item 穿过不同电位区的界面时运动方向改变。
\end{itemize}
相应地,磁场$\sim$各向异性介质,详见\cite{textbook}. 

定量分析,电子的运动方程可写为作用量形式:
\begin{equation}
	\delta\int_\mathcal{L} \mu ds = 0
\end{equation}
其中$\mu$为广义动量沿轨迹$\mathcal{L}$的切向分量;此式与几何光学中的费马原理一致。

在非相对论极限下,作为例子,进一步考虑纯静电势$\phi$,略去比例系数,有\cite{textbook}:
\begin{equation}
	\mu \simeq \phi^{1/2}
\end{equation}
$\mu$即电子光学折射率;与光学折射率$n$相比,$\mu$可变范围广、在空间中连续分布,但分布形式受麦克斯韦方程的约束,不可任取。

在旋转对称、傍轴近似下,电子轨迹方程的解完全类似与几何光学光线。由此构造扫描电镜的电子光学系统。

综上可见,扫描电镜二次电子成像的大致过程为,首先利用电子光学系统使电子束汇聚为充分细的“探针”,偏转电子束方向使电子探针在样品表面扫描,收集对应点逸出的二次电子,进一步分析得到样品表面的形貌。

由此可见,探针的尺度与成像质量密切相关。在其他条件最优化的前提下,探针尺度越纤细,像的分辨程度越高。

%%%%%%%%%%%%%%%%%%%%%%%%%%%%%%%%%%%%%%%% Experiment %%%%%%%%%%%%%%%%%%%%%%%%%%%%%%%%%%%%%%%%
\newpage
\section{实验} \label{experiment}%------------------------------
\subsection{实验仪器}\label{sub:instruments}
\begin{itemize}
\item{\textbf{扫描电子显微镜}}

与光学显微镜不同,扫描电子显微镜中不仅需要电子束替代光学显微镜中的光束,而且是依靠聚焦电子束与样品表面相互作用产生的信号,经收集处理来成像的.它由三部分造成:
\begin{enumerate}[1.]
    \item 电子光学系统,包括电子枪、电磁透镜、扫描线圈等.
    \item 样品室.
    \item 样品所产生信号的收集、处理和显示系统.
\end{enumerate}
\item{\textbf{扩散泵-机械泵机组}

获得真空,工作时镜内真空约为$1.33\times10^{-3}\si{\Pa}$,标称分辨率为$\SI{6.0}{\nano m}$},放大倍率可在15$\sim$25$\times10^4$倍范围内变化.
\end{itemize}
\begin{comment}
%------------------------------------------------------------
\subsection{简要实验步骤}\label{sub:ExperimentalSteps}
分为以下几个步骤:\\\\
\circled{1}熟悉控制和处理程序\\\\
\circled{2}得到石墨的原子分辨像\\\\
\circled{3}观察金膜表面\\\\
\circled{4}标定x、y陶瓷的压电系数\\\\
\end{comment}
%%%%%%%%%%%%%%%%%%%%%%%%%%%%%%%%%%%%%%%% Results & Discussions %%%%%%%%%%%%%%%%%%%%%%%%%%%%%%%%%%%%%%%%
\newpage
\section{结果及讨论}
%------------------------------------------------------------
\subsection{10k倍时不同聚光镜电流成像结果}\label{sub:1}

\begin{table}[htbp]
    \caption{10k倍时不同聚光镜电流对应工作参数}\label{table:1.1}
    \centering
    \begin{tabular}{>{\centering\arraybackslash}p{3cm}%
    >{\centering\arraybackslash}p{2cm}%
    >{\centering\arraybackslash}p{3cm}%
    >{\centering\arraybackslash}p{2cm}%
    >{\centering\arraybackslash}p{2cm}}
    \toprule\toprule
    \textbf{聚光镜电流i} & \textbf{物镜电流} & \textbf{工作距离/mm} & \textbf{亮度} & \textbf{对比度}\\
    \midrule
    400 & 519.03 & 12.0 & -5.7 & 65.1 \\
    450 & 516.50 & 12.0 & -3.7 & 64.7 \\
    500 & 516.78 & 12.0 & -3.9 & 65.1 \\
    550 & 515.70 & 12.0 & 0.0 & 65.9 \\
    600 & 514.56 & 12.0 & -2.5 & 77.3\\
    650  & 513.36 & 12.0 & -1.6 & 82.4 \\
    \bottomrule\bottomrule
\end{tabular}
\end{table}

\begin{figure}[H]
 \centering
 \caption{10k倍,聚光镜电流$i=400$}
 \includegraphics[height=6.75cm, width=9cm]{pictures/400/10k.bmp}
 \label{result:fig1}
\end{figure}

\begin{figure}[H]
 \centering
 \caption{10k倍,聚光镜电流$i=450$}
 \includegraphics[height=6.75cm, width=9cm]{pictures/450/10k.bmp}
 \label{result:fig2}
\end{figure}

\begin{figure}[H]
 \centering
 \caption{10k倍,聚光镜电流$i=500$}
 \includegraphics[height=6.75cm, width=9cm]{pictures/500/10k.bmp}
 \label{result:fig3}
\end{figure}

\begin{figure}[H]
 \centering
 \caption{10k倍,聚光镜电流$i=550$}
 \includegraphics[height=6.75cm, width=9cm]{pictures/550/10k.bmp}
 \label{result:fig4}
\end{figure}

\begin{figure}[H]
 \centering
 \caption{10k倍,聚光镜电流$i=600$}
 \includegraphics[height=6.75cm, width=9cm]{pictures/600/10k.bmp}
 \label{result:fig5}
\end{figure}

\begin{figure}[H]
 \centering
 \caption{10k倍,聚光镜电流$i=650$}
 \includegraphics[height=6.75cm, width=9cm]{pictures/650/10k.bmp}
 \label{result:fig6}
\end{figure}
从上述图片可以看出,随着聚光镜电流增加,10k倍数的图像分辨率也逐渐增加,在聚光镜电流i=600左右最清晰.
%------------------------------------------------------------
\subsection{1k倍时不同聚光镜电流成像结果}\label{sub:2}
\begin{table}[htbp]
    \caption{1k倍时不同聚光镜电流对应工作参数}\label{table:1.1}
    \centering
    \begin{tabular}{>{\centering\arraybackslash}p{3cm}%
    >{\centering\arraybackslash}p{2cm}%
    >{\centering\arraybackslash}p{3cm}%
    >{\centering\arraybackslash}p{2cm}%
    >{\centering\arraybackslash}p{2cm}}
    \toprule\toprule
    \textbf{聚光镜电流i} & \textbf{物镜电流} & \textbf{工作距离/mm} & \textbf{亮度} & \textbf{对比度}\\
    \midrule
    400 & 519.08 & 12.0 & -1.8 & 65.1 \\
    450 & 516.50 & 12.0 & -2.1 & 64.7 \\
    500 & 514.45 & 12.0 & -3.5 & 64.7 \\
    550 & 515.70 & 12.0 & 0.0 & 65.9 \\
    600 & 514.56 & 12.0 & -1.4 & 71.4\\
    650  & 513.36 & 12.0 & -1.6 & 82.4 \\
    \bottomrule\bottomrule
\end{tabular}
\end{table}

\begin{figure}[H]
 \centering
 \caption{1k倍,聚光镜电流$i=400$}
 \includegraphics[height=6.75cm, width=9cm]{pictures/400/1k.bmp}
 \label{result:fig7}
\end{figure}

\begin{figure}[H]
 \centering
 \caption{1k倍,聚光镜电流$i=450$}
 \includegraphics[height=6.75cm, width=9cm]{pictures/450/1k.bmp}
 \label{result:fig8}
\end{figure}

\begin{figure}[H]
 \centering
 \caption{1k倍,聚光镜电流$i=500$}
 \includegraphics[height=6.75cm, width=9cm]{pictures/500/1k.bmp}
 \label{result:fig9}
\end{figure}

\begin{figure}[H]
 \centering
 \caption{1k倍,聚光镜电流$i=550$}
 \includegraphics[height=6.75cm, width=9cm]{pictures/550/1k.bmp}
 \label{result:fig10}
\end{figure}

\begin{figure}[H]
 \centering
 \caption{1k倍,聚光镜电流$i=600$}
 \includegraphics[height=6.75cm, width=9cm]{pictures/600/1k.bmp}
 \label{result:fig11}
\end{figure}

\begin{figure}[H]
 \centering
 \caption{1k倍,聚光镜电流$i=650$}
 \includegraphics[height=6.75cm, width=9cm]{pictures/650/1k.bmp}
 \label{result:fig12}
\end{figure}
从上述图片可以看出,随着聚光镜电流增加,1k倍数的图像分辨率先增加后减少,在聚光镜电流i=500左右最清晰.
%------------------------------------------------------------
\subsection{聚光镜电流为550时不同工作距离成像结果(对景深影响)}\label{sub:3}
\begin{table}[htbp]
    \caption{聚光镜电流为550时不同工作距离对应工作参数}\label{table:1.1}
    \centering
    \begin{tabular}{>{\centering\arraybackslash}p{3cm}%
    >{\centering\arraybackslash}p{2cm}%
    >{\centering\arraybackslash}p{3cm}%
    >{\centering\arraybackslash}p{2cm}%
    >{\centering\arraybackslash}p{2cm}}
    \toprule\toprule
    \textbf{聚光镜电流i} & \textbf{物镜电流} & \textbf{工作距离/mm} & \textbf{亮度} & \textbf{对比度}\\
    \midrule
    550 & 393.66 & 28.0 & +1.4 & 65.1 \\
    550 & 659.69 & 5.0 & +1.4 & 65.1 \\
    \bottomrule\bottomrule
\end{tabular}
\end{table}

\begin{figure}[H]
 \centering
 \caption{聚光镜电流$i=550$,工作距离$28.0\si{\macrometer}$}
 \includegraphics[height=6.75cm, width=9cm]{pictures/WorkingDistance/28.bmp}
 \label{result:fig13}
\end{figure}

\begin{figure}[H]
 \centering
 \caption{聚光镜电流$i=550$,工作距离$5.0\si{\macrometer}$}
 \includegraphics[height=6.75cm, width=9cm]{pictures/WorkingDistance/5.bmp}
 \label{result:fig14}
\end{figure}
从上述图片可以看出,同样是聚焦在样品上,工作距离为$28.0\si{mm}$时景深更深,能更清晰地看到衬底.可以认为这是因为因为样品到衬底距离一定,相对于$28.0\si{mm}$工作距离更可以忽略,所以看清楚样品的时候时候也能较清晰地看到衬底.
%add more subsections for other block
%%%%%%%%%%%%%%%%%%%%%%%%%%%%%%%%%%%%%%%% Conclusion %%%%%%%%%%%%%%%%%%%%%%%%%%%%%%%%%%%%%%%%
\section{结论}\label{conclusions}
本实验探究了SEM的基本构造和原理,并进一步考察了聚光镜电流$i$对分辨率的影响.发现在一定范围内聚光镜电流越大,成像质量越好.实验结果表明,在一定范围内,可以找到最优分辨率,但同时噪声也越显著。对于本仪器(中科科仪KYKY-EM3200型SEM)而言,本文给出建议值$i \approx 500\sim 600$. 

%%%%%%%%%%%%%%%%%%%%%%%%%%%%%%%%%%%%%%%% Questions %%%%%%%%%%%%%%%%%%%%%%%%%%%%%%%%%%%%%%%%
\begin{comment}
\section{实验报告思考题}\label{questions}
\subsection{在$a=23.0mm$、$b=10.0mm$的矩形波导管中能不能传播$\lambda=2cm$、$3cm$和$5cm$的微波?各能传播哪些波型?}\label{sub:question1}
答:根据
\begin{equation}
    \lambda_c=\frac{2}{\sqrt{(m/a)^2+(n/b)^2}}
\end{equation}
我们可以算出可传播的最大波长为$\lambda_{max}=18.3mm$,显然不能传播$\lambda=2cm$、$3cm$和$5cm$的微波,可传输波长在$\lambda_{max}=18.3mm$以下,满足$\lambda_c=\frac{2}{\sqrt{(m/a)^2+(n/b)^2}}$的波长的波型\\

\end{comment}

%%%%%%%%%%%%%%%%%%%%%%%%%%%%%%%%%%%%%%%% Acknowledgements %%%%%%%%%%%%%%%%%%%%%%%%%%%%%%%%%%%%%%%%

\section{致谢}\label{acknowledgments}
感谢老师在实验中的的悉心指导.

\begin{comment}
%%%%%%%%%%%%%%%%%%%%%%%%%%%%%%%%%%%%%%%% Appendix %%%%%%%%%%%%%%%%%%%%%%%%%%%%%%%%%%%%%%%%
\appendix
\section{代码}\label{sub:app.code}
请在附录\ref{sub:app.code}中添加代码。请使用如下Scala的语法高亮描述方法。
\begin{scala}
class TopIO extends Bundle() {
	val boot = Input(Bool()) 
// imem and dmem interface for Tests
	val test_im_wr		= Input(Bool())
	val test_im_rd 		= Input(Bool())
	val test_im_addr 	= Input(UInt(32.W))
	val test_im_in 		= Input(UInt(32.W))
	val test_im_out 	= Output(UInt(32.W))

	val test_dm_wr		= Input(Bool())
	val test_dm_rd 		= Input(Bool())
	val test_dm_addr 	= Input(UInt(32.W))
	val test_dm_in 		= Input(UInt(32.W))
	val test_dm_out 	= Output(UInt(32.W))

	val valid			= Output(Bool())
}
class Top extends Module() {
	val io 		= IO(new TopIO())//in chisel3, io must be wrapped in IO(...) 
	//...
	when (io.boot & io.test_im_wr){
		imm(io.test_im_addr) := io.test_im_in
		} .elsewhen (io.boot & io.test_dm_wr){
		// please finish it
		} //...
}
\end{scala}
\newpage
\end{comment}
%%%%%%%%%%%%%%%%%%%%%%%%%%%%%%%%%%%%%%%% REFERENCE %%%%%%%%%%%%%%%%%%%%%%%%%%%%%%%%%%%%%%%%
\bibliographystyle{unsrt}
\bibliography{Ref}
\nocite{*}


\end{document}

